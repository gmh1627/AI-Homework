%!TEX program = xelatex
\documentclass{article}
\usepackage{amsthm,amsmath,amssymb}
\usepackage[UTF8]{ctex}
\usepackage[tc]{titlepic}
\usepackage{titlesec}
\usepackage{cite}
\usepackage{fancyhdr}
\usepackage{booktabs}
\usepackage{graphicx}
\usepackage{subfigure}
\usepackage{float}
\usepackage{geometry}
\usepackage[section]{placeins}
\usepackage{makeidx}
\usepackage{mathrsfs}
\usepackage{color}
\usepackage{ulem}
\usepackage{enumitem}
\geometry{a4paper,scale=0.8}
\pagestyle{fancy}

\usepackage{hyperref}
\hypersetup{hypertex=true, colorlinks=true, linkcolor=blue, anchorcolor=blue, citecolor=blue}

\usepackage{listings}
\lstset{
    language=Python,
    basicstyle=\small\ttfamily,
    keywordstyle=\color{blue},
    commentstyle=\color{green},
    stringstyle=\color{red},
    showstringspaces=false,
    breaklines=true,
}

\lhead{第 2 次作业\\\today}
\chead{中国科学技术大学\\	DS4001 人工智能原理与技术}

\rhead{Homework 2\\ {\CTEXoptions[today=old]\today}}
\newcommand{\upcite}[1]{\textsuperscript{\cite{#1}}}

\titleformat*{\section}{\bfseries\Large}
\titleformat*{\subsection}{\bfseries\large}

\title{\bfseries 第二次作业(强化学习)}
\author{xxx  \quad  PB22000000}

\begin{document}
\maketitle
\textcolor{red}{\textbf{本次作业需独立完成,不允许任何形式的抄袭行为,如被发现会有相应惩罚。在上方修改你的姓名学号,说明你同意本规定。}}
% \clearpage

\section*{问题1:热身(10分)}
\subsection*{a.计算(5分)}
\begin{table}[h]
    \centering
    \begin{tabular}{|c|c|c|c|c|}
    \hline
    $i$ & $s=0$ & $s=1$ & $s=2$ & $s=3$ \\
    \hline
    0 & 0 & 0 & 0 & 0 \\
    1 & $V^{(1)}(0)$ & $V^{(1)}(1)$ & $V^{(1)}(2)$ & $V^{(1)}(3)$ \\
    2 & $V^{(2)}(0)$ & $V^{(2)}(1)$ & $V^{(2)}(2)$ & $V^{(2)}(3)$ \\
    \hline
    \end{tabular}
    \caption{Value Iteration for $i \in \{0, 1, 2\}$}
    \end{table}

\subsection*{b.计算(5分)}
\textcolor{red}{TODO}




\section*{问题2:Q-Learning(15分)}
\subsection*{a.回答问题(2分)}
\textcolor{red}{TODO} 

\subsection*{b.计算(8分)}
\textcolor{red}{TODO}

\subsection*{c.回答问题(5分)}
\textcolor{red}{TODO}


\section*{问题3:Gobang Programming(55分)}
\subsection*{a.回答问题(2分)}
\textcolor{red}{TODO}

\subsection*{b.代码填空(33分)}
\textcolor{red}{TODO} 
\begin{lstlisting}
class Gobang(UtilGobang):
    
    def get_next_state(self, action: Tuple[int, int, int], noise: Tuple[int, int, int]) -> np.array:

        # BEGIN_YOUR_CODE (our solution is 3 line of code, but don't worry if you deviate from this)
        
        # END_YOUR_CODE

        if noise is not None:
            white, x_white, y_white = noise
            next_state[x_white][y_white] = white
        return next_state

    def sample_noise(self) -> Union[Tuple[int, int, int], None]:

        if self.action_space:
            # BEGIN_YOUR_CODE (our solution is 2 line of code, but don't worry if you deviate from this)
            
            # END_YOUR_CODE
            return 2, x, y
        else:
            return None

    def get_connection_and_reward(self, action: Tuple[int, int, int],
                                  noise: Tuple[int, int, int]) -> Tuple[int, int, int, int, float]:

        # BEGIN_YOUR_CODE (our solution is 4 line of code, but don't worry if you deviate from this)
        
        # END_YOUR_CODE

        return black_1, white_1, black_2, white_2, reward

    def sample_action_and_noise(self, eps: float) -> Tuple[Tuple[int, int, int], Tuple[int, int, int]]:

        # BEGIN_YOUR_CODE (our solution is 8 line of code, but don't worry if you deviate from this)

        # END_YOUR_CODE
        return action, self.sample_noise()

    def q_learning_update(self, s0_: np.array, action: Tuple[int, int, int], s1_: np.array, reward: float,
                          alpha_0: float = 1):

        s0, s1 = self.array_to_hashable(s0_), self.array_to_hashable(s1_)
        self.s_a_visited[(s0, action)] = 1 if (s0, action) not in self.s_a_visited else \
            self.s_a_visited[(s0, action)] + 1
        alpha = alpha_0 / self.s_a_visited[(s0, action)]

        # BEGIN_YOUR_CODE (our solution is 18 line of code, but don't worry if you deviate from this)
        
        # END_YOUR_CODE


\end{lstlisting}



\subsection*{c.结果复现(10分)}
\textcolor{red}{TODO}
你需要将复现结果的截图粘贴在这里。
\begin{figure}[H]
    \centering
    \includegraphics[width=5cm]{pics/temp.jpg}
    \caption{复现结果}
    \label{fig:indesirable_solution}
\end{figure}
图~\ref{fig:indesirable_solution}是一张示例图片,请你按照示例插入图片以及文字叙述。

\subsection*{d.回答问题(10分)}
\textcolor{red}{TODO}

\section*{问题4:Deeper Understanding(10分)}
\subsection*{a.回答问题(5分)}
\textcolor{red}{TODO}

\subsection*{b.回答问题(5分)}
\textcolor{red}{TODO}


\section*{反馈(10分)}
在每次实验报告的最后欢迎反馈你上这门课的感受,你可以写下任何反馈,包括但不限于以下几个方面:课堂、作业难度和工作量、助教工作等等。

\textcolor{red}{TODO} % 你可以先注释掉上面那段话
\begin{itemize}
    \item ......
    \item ......
\end{itemize}

\end{document}